\section*{Introduction}

FleCSI is a compile-time configurable framework designed to support
multi-physics application development.
As such, FleCSI provides a very general set of infrastructure design
patterns that can be specialized and extended to suit the needs of a
broad variety of solver and data requirements.
FleCSI currently supports multi-dimensional mesh topology, geometry, and
adjacency information, as well as n-dimensional hashed-tree data
structures, graph partitioning interfaces, and dependency closures.

FleCSI introduces a functional programming model with control,
execution, and data abstractions that are consistent both with MPI and
with state-of-the-art, task-based runtimes such as Legion and Charm++.
The abstraction layer insulates developers from the underlying runtime,
while allowing support for multiple runtime systems including
conventional models like asynchronous MPI.

The intent is to provide developers with a concrete set of user-friendly
programming tools that can be used now, while allowing flexibility in
choosing runtime implementations and optimizations that can be applied
to future architectures and runtimes.

FleCSI's control and execution models provide formal nomenclature for
describing poorly understood concepts such as kernels and tasks.
FleCSI's data model provides a low-buy-in approach that makes it an
attractive option for many application projects, as developers are not
locked into particular layouts or data structure representations.

\subsection*{Structure}

This document is intended to introduce users to the FleCSI programming
system by providing an overview and examples of the primary programming
model and its components.

% vim: set tabstop=2 shiftwidth=2 expandtab fo=cqt tw=72 :
